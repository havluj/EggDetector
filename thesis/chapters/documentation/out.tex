\chapter{Dokumentace API knihovny}
\section{Package org.cvut.havluja1.eggdetector}{
\label{org.cvut.havluja1.eggdetector}\hskip -.05in
\hbox to \hsize{\textit{ Package Contents\hfil Page}}
\vskip .13in
\hbox{{Classes}}
\entityintro{EggDetector}{org.cvut.havluja1.eggdetector.EggDetector}
\entityintro{SequenceClassifier}{org.cvut.havluja1.eggdetector.SequenceClassifier}
\vskip .1in
\vskip .1in
\subsection{\label{org.cvut.havluja1.eggdetector.EggDetector}Class EggDetector}{
\vskip .1in 
\subsubsection{Count the number of eggs in given images}

The egg detector is a library that helps you count the number of eggs in a given folder.
Egg detector works by using TeonsorFlow Object Detection API in the background. To learn more, see {https://www.tensorflow.com}.
\vskip .1in
\noindent Example usage:
\begin{lstlisting}
EggDetector eggDetector = new EggDetector();
SequenceClassifier sequenceClassifier = eggDetector.evaluate(new File("image_dir"));
System.out.println("final count: " + sequenceClassifier.getFinalCount());
System.out.println("individual scores: " + sequenceClassifier.getIndividualCounts());
eggDetector.closeSession();
\end{lstlisting}
\vskip .1in 
\subsubsection{Declaration}{
\begin{lstlisting}[frame=none]
public class EggDetector
 extends java.lang.Object\end{lstlisting}
\subsubsection{Constructor summary}{
\begin{verse}
{\textit{EggDetector()}}\\
Constructor loads the pre-trained frozen graph into memory.
\end{verse}
}
\subsubsection{Method summary}{
\begin{verse}
{\textit{closeSession()}}\\
Closes the EggDetector session.

{\textit{evaluate(File)}}\\
Runs egg detection on a given \textit{ dir}.

{\textit{getMinimalConfidence()}}\\
Get the minimalConfidence setting for this instance.


{\textit{isDebugMode()}}\\
Get this instance's debug mode setting.

{\textit{setDebugMode(boolean)}}\\
Set this instance's debug mode setting.

{\textit{setMinimalConfidence(float)}}\\
Set the minimalConfidence setting for this instance.

{\textit{toString()}} \\
\end{verse}
}
\subsubsection{Constructors}{
\vskip -2em
\begin{itemize}
\item{ 
\index{EggDetector()}
{\textit{EggDetector}}\\
\begin{lstlisting}[frame=none]
public EggDetector()\end{lstlisting} %end signature
\begin{itemize}
\item{
{\bf  Description}



Constructor loads the pre-trained frozen graph into memory.

It also checks whether TensorFlow is supported on your platform.
}
\end{itemize}
}%end item
\end{itemize}
}
\subsubsection{Methods}{
\vskip -2em
\begin{itemize}
\item{ 
\index{closeSession()}
{\textit{closeSession}}\\
\begin{lstlisting}[frame=none]
public void closeSession() throws java.lang.IllegalStateException\end{lstlisting} %end signature
\begin{itemize}
\item{
{\bf  Description}

Closes the EggDetector session. This instance of EggDetector will not be usable again.
}
\item{{\bf  Throws}
  \begin{itemize}
   \item{\vskip -.6ex \texttt{java.lang.IllegalStateException} -- if the session has been closed already by calling \texttt{\small closeSession()}{\small 
\refdefined{org.cvut.havluja1.eggdetector.EggDetector.closeSession()}}}
  \end{itemize}
}%end item
\end{itemize}
}%end item
\item{ 
\index{evaluate(File)}
{\textit{evaluate}}\\
\begin{lstlisting}[frame=none]
public SequenceClassifier evaluate(java.io.File dir) throws java.lang.IllegalArgumentException, java.lang.IllegalStateException\end{lstlisting} %end signature
\begin{itemize}
\item{
{\bf  Description}



Runs egg detection on a given \textit{ dir}.
}
\item{
{\bf  Parameters}
  \begin{itemize}
   \item{
\texttt{dir} -- a directory containing .jpg or .png files for object detection}
  \end{itemize}
}%end item
\item{{\bf  Returns} -- 
 
}%end item
\item{{\bf  Throws}
  \begin{itemize}
   \item{\vskip -.6ex \texttt{java.lang.IllegalArgumentException} -- if \textit{ dir} is not a directory or contains no images}
   \item{\vskip -.6ex \texttt{java.lang.IllegalStateException} -- if the session has been closed already by calling \texttt{\small closeSession()}{\small 
\refdefined{org.cvut.havluja1.eggdetector.EggDetector.closeSession()}}}
  \end{itemize}
}%end item
\end{itemize}
}%end item
\item{ 
\index{getMinimalConfidence()}
{\textit{getMinimalConfidence}}\\
\begin{lstlisting}[frame=none]
public float getMinimalConfidence()\end{lstlisting} %end signature
\begin{itemize}
\item{
{\bf  Description}



Get the \textit{minimalConfidence} setting for this instance.

Minimal Confidence score is used as a confidence boundary during the process of object detection. An object that has been detected with a confidence score lower than \textit{ minimalConfidence} is ignored. An object that has been detected with a confidence score higher or equal than \textit{ minimalConfidence} is added to the final result list.
}
\item{{\bf  Returns} -- 
This instance's \textit{minimalConfidence} setting. 
}%end item
\end{itemize}
}%end item
\item{ 
\index{isDebugMode()}
{\textit{isDebugMode}}\\
\begin{lstlisting}[frame=none]
public boolean isDebugMode()\end{lstlisting} %end signature
\begin{itemize}
\item{
{\bf  Description}



Get this instance's debug mode setting.

If debug mode is enabled (set to true), the library will open a \texttt{\small JFrame}{\small 
\refdefined{javax.swing.JFrame}} for each processed image with detections graphically highlighted.
}
\item{{\bf  Returns} -- 
debug mode setting for this instance 
}%end item
\end{itemize}
}%end item
\item{ 
\index{setDebugMode(boolean)}
{\textit{setDebugMode}}\\
\begin{lstlisting}[frame=none]
public void setDebugMode(boolean debugMode) throws java.lang.IllegalStateException\end{lstlisting} %end signature
\begin{itemize}
\item{
{\bf  Description}



Set this instance's debug mode setting.

If debug mode is enabled (set to true), the library will open a \texttt{\small JFrame}{\small 
\refdefined{javax.swing.JFrame}} for each processed image with detections graphically highlighted.
}
\item{
{\bf  Parameters}
  \begin{itemize}
   \item{
\texttt{debugMode} -- turn the debug mode on or off}
  \end{itemize}
}%end item
\item{{\bf  Throws}
  \begin{itemize}
   \item{\vskip -.6ex \texttt{java.lang.IllegalStateException} -- if the session has been closed already by calling \texttt{\small closeSession()}{\small 
\refdefined{org.cvut.havluja1.eggdetector.EggDetector.closeSession()}}}
  \end{itemize}
}%end item
\end{itemize}
}%end item
\item{ 
\index{setMinimalConfidence(float)}
{\textit{setMinimalConfidence}}\\
\begin{lstlisting}[frame=none]
public void setMinimalConfidence(float minimalConfidence) throws java.lang.IllegalStateException\end{lstlisting} %end signature
\begin{itemize}
\item{
{\bf  Description}



Set the \textit{minimalConfidence} setting for this instance.

Minimal Confidence score is used as a confidence boundary during the process of object detection. An object that has been detected with a confidence score lower than \textit{minimalConfidence} is ignored. An object that has been detected with a confidence score higher or equal than \textit{minimalConfidence} is added to the final result list.
}
\item{
{\bf  Parameters}
  \begin{itemize}
   \item{
\texttt{minimalConfidence} -- \textit{ minimalConfidence} for this instance}
  \end{itemize}
}%end item
\item{{\bf  Throws}
  \begin{itemize}
   \item{\vskip -.6ex \texttt{java.lang.IllegalStateException} -- if the session has been closed already by calling \texttt{\small closeSession()}{\small 
\refdefined{org.cvut.havluja1.eggdetector.EggDetector.closeSession()}}}
  \end{itemize}
}%end item
\end{itemize}
}%end item
\item{ 
\index{toString()}
{\textit{toString}}\\
\begin{lstlisting}[frame=none]
public java.lang.String toString()\end{lstlisting} %end signature
}%end item
\end{itemize}
}
}
\subsection{\label{org.cvut.havluja1.eggdetector.SequenceClassifier}Class SequenceClassifier}{
\vskip .1in 
\subsubsection{A class containing object detection results for a given directory}

SequenceClassifier is a data class containing the results of object detection for a given directory. When constructed, object detection is performed on all images and results are stored in memory.
\vskip .1in
\noindent Example usage:
\begin{lstlisting}
EggDetector eggDetector = new EggDetector();
SequenceClassifier sequenceClassifier = eggDetector.evaluate(new File("image_dir"));
System.out.println("final count: " + sequenceClassifier.getFinalCount());
System.out.println("individual scores: " + sequenceClassifier.getIndividualCounts());
eggDetector.closeSession();
\end{lstlisting}
\vskip .1in 
\subsubsection{Declaration}{
\begin{lstlisting}[frame=none]
public class SequenceClassifier
 extends java.lang.Object\end{lstlisting}
\subsubsection{Method summary}{
\begin{verse}
{\textit{getFinalCount()}}\\
Get the final score for the entire directory.

{\textit{getIndividualCounts()}}\\
Gets the individual egg count for every image provided.
\end{verse}
}
\subsubsection{Methods}{
\vskip -2em
\begin{itemize}
\item{ 
\index{getFinalCount()}
{\textit{getFinalCount}}\\
\begin{lstlisting}[frame=none]
public java.lang.Integer getFinalCount()\end{lstlisting} %end signature
\begin{itemize}
\item{
{\bf  Description}



Get the final score for the entire directory.

The final score is calculated as follows:\begin{itemize}
\item{\vskip -.8ex individual scores of images are sorted and counted}
\item{\vskip -.8ex the highest egg count is returned as a result if we detected this egg count in at least two different images}
\item{\vskip -.8ex if no two images contain the same egg count, the highest detected egg count is returned}
\item{\vskip -.8ex if no eggs are detected in any of the images, 0 is returned}
\end{itemize}

}
\item{{\bf  Returns} -- 
final egg count for this instance 
}%end item
\end{itemize}
}%end item
\item{ 
\index{getIndividualCounts()}
{\textit{getIndividualCounts}}\\
\begin{lstlisting}[frame=none]
public java.util.Map getIndividualCounts()\end{lstlisting} %end signature
\begin{itemize}
\item{
{\bf  Description}

Gets the individual egg count for every image provided.
}
\item{{\bf  Returns} -- 
A map of individual scores. The key is the filename. The value is the egg count. 
}%end item
\end{itemize}
}%end item
\end{itemize}
}
}
}
