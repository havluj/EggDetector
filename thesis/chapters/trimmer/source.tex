\chapter{Folder Trimmer}\label{chap:folder_trimmer}
Tato p��loha obsahuje zdrojov� k�d a zp�sob pou�it� n�stroje \texttt{FolderTrimmer}. Kompletn� dokumentace a zdrojov� k�d je dostupn� na p�ilo�en�m m�diu (viz p��loha \ref{chap:cd}).

\section{Dokumentace (v AJ)}
Folder trimmer is a command line tool for deleting useless data the download script downloads.

\subsection{What does it do}
\begin{itemize}
	\item Deletes files that do not end on either \texttt{.png}, \texttt{.xml}, or \texttt{.txt}.
	\item If a folder does not contain another folder or at least one of the files listed above, the folder is deleted.
\end{itemize}

\subsection{Usage manual}
\begin{itemize}
	\item Compile (if needed) the app with \texttt{mvn clean package}.
	\item Run with \texttt{target/run.sh} on Linux or \texttt{target\textbackslash run.bat} on Windows. Both scripts accept two parameters:
	
	\begin{itemize}
		\item First parameter can be \texttt{true} or \texttt{false} and it determines whether you want to delete everything except manually tagged data (containing imgdata.xml) in case of \texttt{true} or if you just want to delete the useless data in case of \texttt{false}.
		\item Second parameter specifies the root folder of the data that should be trimmed. 
		\item For example to delete only useless data but keep all the folders with	some image data, use:
		\begin{lstlisting}[language=bash] 
		target/run.sh false ~/eggdetector/data
		\end{lstlisting}
		\item To delete everything except tagged data, run: 
		\begin{lstlisting}[language=bash] 
		target/run.sh true ~/eggdetector/data
		\end{lstlisting}
	\end{itemize}
\end{itemize}

\section{Zdrojov� k�d}
\begin{lstlisting}[basicstyle=\tiny,caption={Zdrojov� k�d n�stroje FolderTrimmer}]
package org.cvut.havluja1.foldertrimmer;

import java.io.File;
import java.io.IOException;

import org.apache.commons.io.FileUtils;
import org.apache.commons.io.FilenameUtils;

public class FolderTrimmer {
	public static void main(String[] args) throws IOException {
		File rootDir = new File(args[0]);
		
		if (!rootDir.exists() || !rootDir.isDirectory()) {
			throw new IllegalArgumentException("root dir does not exist");
		}
		
		System.out.println("finding useless data...");
		findAndDeleteEmptyDirs(rootDir);
		System.out.println("done");
	}
	
	private static void findAndDeleteEmptyDirs(File dir) {
		final boolean[] shouldBeDeleted = {true};
		final boolean leaveOnlyTaggedData = System.getProperty("leaveonlytagged").equalsIgnoreCase("true");
		
		File[] toBeProcessed = dir.listFiles((file, s) -> {
			File workingFile = new File(file, s);
			
			// if dir -> return true and tag this folder not to be deleted
			if (workingFile.isDirectory()) {
				shouldBeDeleted[0] = false;
				return true;
			}
	
	
			if (workingFile.isFile()) {
				if (leaveOnlyTaggedData) { // if we want to keep only tagged data
					if (s.equals("imgdata.xml")) {
						shouldBeDeleted[0] = false;
						return false;
					} else {
						if (FilenameUtils.getExtension(s).equals("png")) {
							return false;
						}
						return true;
					}
				} else { // If file and is not xml, txt or png return true. If it is, tag this folder not to be deleted.
					if (FilenameUtils.getExtension(s).equals("xml")
						|| FilenameUtils.getExtension(s).equals("png")
						|| FilenameUtils.getExtension(s).equals("txt")) {
						shouldBeDeleted[0] = false;
						return false;
					} else {
						return true;
					}
				}
			}
	
			return true;
		});
	
		if (shouldBeDeleted[0]) {
			try {
				FileUtils.deleteDirectory(dir);
				System.out.println("[D] deleting dir: " + dir.getAbsolutePath());
			} catch (IOException e) {
				e.printStackTrace();
			}
		} else {
			if (toBeProcessed.length > 0) {
				for (File currFile : toBeProcessed) {
					// if file -> delete
					if (currFile.isFile()) {
						if (currFile.delete()) {
							System.out.println("[F] deleting file: " + currFile.getAbsolutePath());
						}
						continue;
					}
					
					// if dir -> recursive call
					if (currFile.isDirectory()) {
						findAndDeleteEmptyDirs(currFile);
					}
				}
			}
		}
	}
}
\end{lstlisting}